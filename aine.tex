% --- Template for thesis / report with tktltiki2 class ---
% 
% last updated 2013/02/15 for tkltiki2 v1.02

\documentclass[finnish]{tktltiki2}

% tktltiki2 automatically loads babel, so you can simply
% give the language parameter (e.g. finnish, swedish, english, british) as
% a parameter for the class: \documentclass[finnish]{tktltiki2}.
% The information on title and abstract is generated automatically depending on
% the language, see below if you need to change any of these manually.
% 
% Class options:
% - grading                 -- Print labels for grading information on the front page.
% - disablelastpagecounter  -- Disables the automatic generation of page number information
%                              in the abstract. See also \numberofpagesinformation{} command below.
%
% The class also respects the following options of article class:
%   10pt, 11pt, 12pt, final, draft, oneside, twoside,
%   openright, openany, onecolumn, twocolumn, leqno, fleqn
%
% The default font size is 11pt. The paper size used is A4, other sizes are not supported.
%
% rubber: module pdftex

% --- General packages ---

\usepackage[utf8]{inputenc}
\usepackage[T1]{fontenc}
\usepackage{lmodern}
\usepackage{microtype}
\usepackage{amsfonts,amsmath,amssymb,amsthm,booktabs,color,enumitem,graphicx}
%\usepackage[pdflatex,hidelinks]{hyperref}
%\usepackage[hidelinks]{hyperref}
\usepackage[colorlinks=false, pdfborder={0 0 0}]{hyperref}


% Automatically set the PDF metadata fields
\makeatletter
\AtBeginDocument{\hypersetup{pdftitle = {\@title}, pdfauthor = {\@author}}}
\makeatother

% --- Language-related settings ---
%
% these should be modified according to your language

% babelbib for non-english bibliography using bibtex
\usepackage[fixlanguage]{babelbib}
\selectbiblanguage{finnish}

% add bibliography to the table of contents
\usepackage[nottoc]{tocbibind}
% tocbibind renames the bibliography, use the following to change it back
\settocbibname{Lähteet}

% --- Theorem environment definitions ---

\newtheorem{lau}{Lause}
\newtheorem{lem}[lau]{Lemma}
\newtheorem{kor}[lau]{Korollaari}

\theoremstyle{definition}
\newtheorem{maar}[lau]{Määritelmä}
\newtheorem{ong}{Ongelma}
\newtheorem{alg}[lau]{Algoritmi}
\newtheorem{esim}[lau]{Esimerkki}

\theoremstyle{remark}
\newtheorem*{huom}{Huomautus}


% --- tktltiki2 options ---
%
% The following commands define the information used to generate title and
% abstract pages. The following entries should be always specified:

\title{Käyttäytymisen muokkaaminen mobiilisovelluksilla}
\author{Topias Heinonen}
\date{\today}
\level{Aine}
\abstract{Ihmiset tekevät usein itselleen ja yhteiskunnalle parempia valintoja, kun heille antaa enemmän tietoa tai motivaatiota.
Tässä kirjoituksessa käyn läpi tutkimusta mobiilisovelluksista, joiden tarkoituksena on antaa sekä tietoa että motivaatiota parempien valintojen tueksi.
Tutkimuksessa keskitytään erityisesti pelillisten elementtien vaikutusta käyttäytymisen muokkaamisessa.
}

% The following can be used to specify keywords and classification of the paper:

\keywords{avainsana 1, avainsana 2, avainsana 3}

% classification according to ACM Computing Classification System (http://www.acm.org/about/class/)
% This is probably mostly relevant for computer scientists
% uncomment the following; contents of \classification will be printed under the abstract with a title
% "ACM Computing Classification System (CCS):"
% \classification{}

% If the automatic page number counting is not working as desired in your case,
% uncomment the following to manually set the number of pages displayed in the abstract page:
%
% \numberofpagesinformation{16 sivua + 10 sivua liitteissä}
%
% If you are not a computer scientist, you will want to uncomment the following by hand and specify
% your department, faculty and subject by hand:
%
% \faculty{Matemaattis-luonnontieteellinen}
% \department{Tietojenkäsittelytieteen laitos}
% \subject{Tietojenkäsittelytiede}
%
% If you are not from the University of Helsinki, then you will most likely want to set these also:
%
% \university{Helsingin Yliopisto}
% \universitylong{HELSINGIN YLIOPISTO --- HELSINGFORS UNIVERSITET --- UNIVERSITY OF HELSINKI} % displayed on the top of the abstract page
% \city{Helsinki}
%


\begin{document}

% --- Front matter ---

\frontmatter      % roman page numbering for front matter

\maketitle        % title page
\makeabstract     % abstract page

\tableofcontents  % table of contents

% --- Main matter ---

\mainmatter       % clear page, start arabic page numbering

\section{Johdanto}

%Hylätty kappale
%Mobiilisovellusten käyttämistä ihmisten käyttäytymisen muokkaamiseen on tutkittu paljon~\cite{houston, ubifit, ubigreen, obesity, movipill, tripzoom, fitocracy}.
%Tässä kirjoituksessa käsittelemäni sovellukset jakautuvat liikuntaa lisääviin ja matkantekoon vaikuttaviin sovelluksiin.

Matkapuhelimet kulkevat nykyään mukana jatkuvasti, ja ne sisältävät runsaasti sensoreita monipuoliseen ympäristön havainnointiin.
Nämä ominaisuudet mahdollistavat matkapuhelimen käytön jokapaikan tietoteknisenä alustana. 
Jokapaikan tietotekniikka (Pervasive Computing) määriteltiin vuonna 1991 tarkoittamaan ympäristöä, jossa tietotekniikka on niin nivoutunut arkeen, että se on muuttunut lähes näkymättömäksi \cite{weiser}.
Käyttäjän ei tarvitse tehtävää tehdessään ajatella käyttämäänsä laitetta, vaan hän voi keskittyä itse tehtävään.
Laitteisto reagoi ennakoivasti muutoksiin käyttäjän tilassa, ilman suoria komentoja \cite{pervasive}.

Vakuuttava teknologia (persuasive technology \cite{fogg1998}) tarkoittaa teknologiaa, jonka tarkoituksena on muokata käyttäjän asenteita tai käyttäytymistä.
Käyttäytymisen muokkaamisen on oltava sovelluksen tarkoituksena, minkä takia esimerkiksi Dance Dance Revolution tai Ingress eivät kuulu tämän kirjoituksen piiriin.
Ne lisäävät käyttäjänsä fyysistä aktiivisuutta, mutta niiden tarkoituksena ei ole motivoida liikkumaan, vaan viihdyttää.

Käyttäytymisen muokkaamisen tavoitteena on saada aikaan pysyviä elämäntapamuutoksia.
Transteoreettinen muutosvaihemalli (transtheoretical model of behaviour change, TTM \cite{ttm}) tarjoaa teoreettisen taustan elämäntapamuutoksille.
Muutosvaihemalli kuvaa elämäntapojen muutoksen pitkäkestoiseksi prosessiksi, joka etenee vaiheittain.
Sovellus voi tukea prosessia tarjoamalla tietoa ja motivaatiota sen eri vaiheissa.

Tässä kirjoituksessa käsittelen ensin muutosvaihemallin teoriaa luvussa 2.
Seuraavaksi esittelen luvuissa 3 ja 4 liikunnan lisäämiseen tarkoitettuja sovelluksia sekä matkantekoon vaikuttavia sovelluksia.
Luvussa 5 tarkastelen pelillistämisen (gamification \cite{gamification}) osuutta sovellusten vakuuttavuudessa.
Perustelen pelillistämisen olevan tässä merkittävä tekijä silloinkin, kun sovelluksen kehittäjät eivät ole ajatelleet sovellustaan pelinä.
Luvussa 6 esitellään tiivistetysti tekstin pääkohdat.

\section{Käyttäytymisen muokkaamisen teoriaa}

Tässä luvussa käsittelen käyttäytymisen muokkaamisen teoreettista taustaa lähinnä transteoreettisen muutosvaihemallin \cite{ttm} kautta.

\section{Matkantekoon vaikuttavat sovellukset}

Tässä kohdassa käsittelen mobiilisovelluksia, joiden tarkoituksena on vaikuttaa käyttäjän matkantekoon. Näitä ovat UbiGreen~\cite{ubigreen} ja tripzoom~\cite{tripzoom}.

\subsection{UbiGreen-Sovellus}

Tutkimusryhmä laati UbiGreen Transportation Display -kännykkäsovelluksen, joka kertoo käyttäjän matkustamisen ympäristöystävällisyydestä. Sovellus pyrkii havaitsemaan käyttäjän tekemiä matkoja ja määrittelemään, mitkä niistä ovat ympäristöystävällisiä. Tutkimusryhmä määritteli ympäristöystävällinen matkan tarkoittavan mitä tahansa muuta matkaa kuin yksin autolla ajettua ja kohteli kaikkia tälläisiä matkoja tasa-arvoisesti.

Aina, kun sovellus havaitsee ympäristöystävällisen matkan, se päivittää puhelimen taustakuvaa. Tutkimusryhmä käytti kahta erilaista taustakuvasarjaa. Toisessa sarjassa oli puu, johon ympäristöystävällisten matkojen kertyessä kasvaa lehtiä, kukkia ja omenoita. Toisessa oli jäärkarhu pienellä jäävuorella. Ympäristöystävällisten matkojen kertyessä jäävuori kasvaa, jäävuoren viereen tulee hylkeitä ja kaloja ruoaksi sekä jääkarhu saa vuorelleen kavereita/perhettä.

Kummankin kuvasarjan viimeinen kuva oli erityinen: puusarjassa puu kasvaa hedelmää, ja jääkarhusarjassa taivaalle ilmestyy revontulet. Jokaisen viikon alussa kuvasarja palautui sarjan ensimmäiseen kuvaan.

Sovellus ei pystynyt havaitsemaan automaattisesti, millä kulkuvälineellä käyttäjä liikkui. Tämän takia sovellus kysyi käyttäjältä havaitun matkan jälkeen, millä kulkuvälineellä matka oli tehty. Jos sovellus ei havainnut matkaa automaattisesti, käyttäjä pystyi raportoimaan matkan manuaalisesti.

\subsection{Kenttätutkimus}
Tutkimusryhmä järjesti kolmen viikon pituisen kenttätutkimuksen, johon osallistui 13 henkilöä. Osallistujien tekemien päivittäisten matkojen määrä vastasi aiempaa tutkimusta, mikä osoittaa sovelluksen keränneen matkatietoa luotettavasti.

Varhaiseksi prototyypiksi UbiGreen tunnisti matkat melko hyvin. Manuaalisesti lisättyjen matkojen osuus kaikista matkoista oli silti korkea (41\%).
Ongelmana oli paitsi matkojen tunnistamisen tekninen vaikeus, myös sovelluksen pitkä viive matkan päättymisen ja tämän tunnistamisen välillä.
Osallistujat lisäsivät joskus matkan manuaalisesti viipeen aikana.
Olisi mielenkiintoista nähdä sama koe uusittuna hyödyntäen uusia menetelmiä matkustustavan havaitsemiseen \cite{hemminki}.

Monet osallistujat mielsivät UbiGreenin peliksi. He rinnastivat ympäristöystävällisten matkojen tekemisen pisteiden ansaitsemiseen ja taustakuvan vaihtumisen uusien tasojen avaamiseen. Jotkut tunsivat houkutusta tehdä ylimääräisiä matkoja saadakseen lisää pisteitä. 

Pelisuunnittelun teorian mukaan uuden löytäminen on yksi pelien koukuttavimpia ominaisuuksia \cite[s.~90]{theoryoffun}\cite[s.~109]{gamedesign}. Tutkimusryhmä päätti olla julkistamatta kuvasarjojen kuvia etukäteen osallistujille [1], mikä toi sovellukseen uuden löytämisen jännityksen ja sai sovelluksen tuntumaan entistä enemmän peliltä. Kun peli ei enää tarjoa uutta, peliin kyllästyy [3 s. 42], mikä näkyy joidenkin osallistujien haluna saada uusia kuvasarjoja jokaiselle viikolle.

Vaikka sovellusta ei oltu suunniteltu sosiaaliseksi, se herätti keskustelua kodeissa ja työpaikoilla. Joidenkin osallistujien työtoverit kiinnostuivat tutkimuksesta niin paljon, että kyselivät osallistujan etenemisestä päivittäin. Lisäksi molemmat tutkimukseen osallistuneet pariskunnat alkoivat kilpailla keskenään “pisteiden” keräämisessä. Tutkimusryhmä miettiikin, voisiko kilpailua ja muita sosiaalisia motivaattoreita hyödyntää laajemmin. Sittemmin Fitocracyn kaltaiset palvelut ovat osoittaneet [5], että näin tosiaankin on.

Osallistujat arvostivat sovelluksen antamaa palautetta. Siltikin, suurin osa koki, ettei se muuttanut heidän aiempia matkustustottumuksiaan. Suurimmalla osalla syynä oli, että he kokivat matkustavansa jo tarpeeksi ympäristöystävällisesti, eikä siten ollut tarvetta muutokselle. Jotkut osallistujista sanoivat myös, että sovelluksen antama visuaalinen palaute ei yksinkertaisesti motivoinut tarpeeksi muuttamaan käytöstä.

\section{Liikuntaan vaikuttavat sovellukset}

Tässä kohdassa käsittelen mobiilisovelluksia, joiden tarkoituksena on korottaa käyttäjän fyysistä aktiivisuutta. Näitä ovat Houston~\cite{houston}, UbiFit Garden~\cite{ubifit} ja Fitocracy~\cite{fitocracy}. Lisäksi käsittelen Arteaga et al. tutkimusta iPhone-peleistä \cite{obesity}..

\section{Pelillistäminen}

Käsittelen Pelillistämistä~\cite{gamification} omana kohtanaan. Käsittelen tässä kohdassa myös lääkkeiden ottamiseen kannustavaa MoviPill-peliä \cite{movipill}.
Sen jälkeen tunnistan pelillisiä elementtejä myös aiemmin esittelemistäni sovelluksista.

\section{Yhteenveto}

% --- References ---
%
% bibtex is used to generate the bibliography. The babplain style
% will generate numeric references (e.g. [1]) appropriate for theoretical
% computer science. If you need alphanumeric references (e.g [Tur90]), use
%
% \bibliographystyle{babalpha-lf}
%
% instead.

\bibliographystyle{babalpha-lf}
\bibliography{references-fi}


% --- Appendices ---

% uncomment the following

% \newpage
% \appendix
% 
% \section{Esimerkkiliite}

\end{document}
